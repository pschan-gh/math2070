\documentclass[a4paper,12pt]{report}
\usepackage[T1]{fontenc}
\usepackage{times}
\usepackage{amsthm}
\usepackage{amscd,amssymb,stmaryrd}
\usepackage{amsmath}
\usepackage{graphicx}
\usepackage{fancybox}
\usepackage{amstext}
\usepackage{color}
\usepackage{mathtools}
\usepackage{pdflscape}
\usepackage{listings}
\usepackage{epic,eepic}
\usepackage{fancyhdr}
\usepackage{hyperref}
\usepackage[capitalise]{cleveref}


\newcommand{\abs}[1]{\left|#1\right|}
\newcommand{\ds}{\displaystyle}
\newcommand{\ol}[1]{\overline{#1}}
\newcommand{\oll}[1]{\overline{\overline{#1}}}
\newcommand{\bs}{\backslash}
\newcommand{\Frac}{\mathrm{Frac}}
\newcommand{\im}{\mathrm{im}\,}
\newcommand{\ZZ}{\mathbb{Z}}
\newcommand{\ra}{\longrightarrow}
\newcommand{\ord}{\mathrm{ord}\,}
\newcommand{\GL}{{\rm GL}}
\newcommand{\SL}{{\rm SL}}
\newcommand{\SO}{{\rm SO}}
\newcommand{\colvec}[1]{\begin{pmatrix}#1\end{pmatrix}}
\newcommand{\Span}{{\rm Span}\,}
\newcommand{\Rank}{{\rm Rank}\,}
\newcommand{\nullity}{{\rm nullity}\,}
\newcommand{\adj}{{\rm adj}\,}
\newcommand{\Proj}{{\rm Proj}}
\newcommand{\ora}{\overrightarrow}
\newcommand{\ve}{\varepsilon}
\newcommand{\phib}{\ol{\phi}}

\newcommand{\class}{}

\renewcommand{\ord}{\mathrm{ord}\,}

\newcounter{statement}
\numberwithin{statement}{chapter}

\newtheorem{thm}[statement]{Theorem}
\newtheorem{prop}[statement]{Proposition}
\newtheorem{defn}[statement]{Definition}
\newtheorem{lemma}[statement]{Lemma}
\newtheorem{claim}[statement]{Claim}
\newtheorem{cor}[statement]{Corollary}
\newtheorem{fact}[statement]{Fact}
\newtheorem{example}[statement]{\bf Example}
\newtheorem{eg}[statement]{\bf Example}
\newtheorem{ex}[statement]{\bf Exercise}
\newtheorem*{notation}{\bf Notation}
\newtheorem*{sol}{\bf Solution}
\newtheorem*{remark}{\bf Remark}
\numberwithin{equation}{chapter}
\numberwithin{section}{chapter}
\numberwithin{subsection}{section}

\renewcommand{\thesubsection}{\arabic{section}.\arabic{subsection}}
\renewcommand{\thesection}{\thechapter.\arabic{section}}
\begin{document}
\title{Math 2070}
\setcounter{chapter}{1}\setcounter{section}{0}
\setcounter{subsection}{0}
\setcounter{statement}{0}

\chapter*{Math 2070 Week 1}
{\bf Topics: }Groups


\quad\\\hrule
\quad\\
\subsection*{Motivation}


\quad\\\hrule
\quad\\

\begin{itemize}
\item 
How many ways are there to color a cube, such that each face is either red or green?




 {\bf Answer:} 
10. Why?




\item 
How many ways are there to form a triangle with three sticks of equal lengths, colored
red, green and blue, respectively?




\item 
What are the symmetries of an equilateral triangle?




\textbf{Dihedral Group $D_3$}










\end{itemize}




\quad\\\hrule
\quad\\
\subsection*{Cayley Table}



\begin{center}
\begin{tabular}{|c|c|c|c|}
\hline
*&$a$&$b$&$c$ \\
\hline
$a$&$a^2$&$ab$&$ac$ \\
\hline
$b$&$ba$&$b^2$&$bc$ \\
\hline
$c$&$ca$&$cb$&$c^2$\\\hline
\end{tabular}
\end{center}



\quad\\\hrule
\quad\\





\textbf{Cayley Table for $D_3$}



\begin{center}
\begin{tabular}{|c|c|c|c|c|c|c|}
\hline

*
&$r_0$&$r_1$&$r_2$&$s_0$&$s_1$&$s_2$ \\
\hline
$r_0$&$r_0$&$r_1$&$r_2$&$s_0$&$s_1$&$s_2$ \\
\hline
$r_1$&$r_1$&$r_2$&$r_0$&$s_1$&$s_2$&$s_0$ \\
\hline
$r_2$&$r_2$&$r_0$&$r_1$&$s_2$&$s_0$&$s_1$ \\
\hline
$s_0$&$s_0$&$s_2$&$s_1$&$r_0$&$r_2$&$r_1$ \\
\hline
$s_1$&$s_1$&$s_0$&$s_2$&$r_1$&$r_0$&$r_2$ \\
\hline
$s_2$&$s_2$&$s_1$&$s_0$&$r_2$&$r_1$&$r_0$\\\hline
\end{tabular}
\end{center}



\quad\\\hrule
\quad\\
\section*{Groups}

\begin{defn}
A group $G$ is a set equipped with a binary operation $*: G \times G \ra G$ (typically called  {\bf  group operation}  or " {\bf multiplication} "), such that:
\begin{itemize}
\item 
 \underline{  {\bf Associativity} } 
\[(a* b)* c = a * (b * c),\]
for all $a, b, c \in G$. In other words, the group operation is  {\bf associative} .

\item 
 \underline{  {\bf Existence of an Identity Element} } 



There is an element $e \in G$, called an  {\bf identity element} , such that:
\[g* e = e * g = g,\]
for all $g \in G$.

\item 
 \underline{  {\bf Invertibility} } 



Each element $g \in G$ has an  {\bf inverse}  $g^{-1} \in G$, such that:
\[g^{-1}* g = g* g^{-1} = e.\]
\end{itemize}
\end{defn}

\begin{itemize}
\item 
Note that we do not require that $a* b = b * a$.
  
\item 
We often write $ab$ to denote $a* b$.
  \end{itemize}

\begin{defn}
If $ab = ba$ for all $a, b \in G$. We say that the group operation is
 {\bf commutative} , and that $G$ is an  {\bf abelian group} .
\end{defn}

\begin{eg}
The following sets are groups, with respect to the specified group operations:
\begin{itemize}
\item 
$G = \mathbb{Q} \backslash \{0\}$, where the group operation is the usual multiplication for rational numbers.
The identity is $e = 1$, and the inverse of $a \in \mathbb{Q}\backslash\{0\}$ is $a^{-1} = \frac{1}{a}$.
The group $G$ is abelian.
  
\item 
$G = \mathbb{Q}$, where the group operation is the usual addition $+$ for rational numbers. The identity is $e = 0$.
The inverse of $a \in \mathbb{Q}$ with respect to $+$ is $-a$.
Note that $\mathbb{Q}$ is  {\bf NOT}  a group with respect to multiplication. For in that case, we have $e = 1$, but $0 \in \mathbb{Q}$ has no inverse
$0^{-1} \in \mathbb{Q}$ such that $0\cdot 0^{-1} = 1$.
  \end{itemize}





\quad\\\hrule
\quad\\

\end{eg}

\begin{ex}
Verify that the following sets are groups under the specified binary operation:
\begin{itemize}
\item 
$(\mathbb{Z}, +)$
  
\item 
$(\mathbb{R}, +)$
  
\item 
$(\mathbb{R}^\times, \cdot)$
  
\item 
$(U_m, \cdot)$, where $m \in \mathbb{N}$,
\[U_m = \{1, \xi_m, \xi_m^2, \ldots, \xi_m^{m - 1}\},\]
and
$\xi_m = e^{2\pi i/m} = \cos(2\pi/m) + i\sin(2\pi/m) \in \mathbb{C}$.


  
\item 
The set of bijective functions $f: \mathbb{R} \ra \mathbb{R}$,
where $f * g:= f \circ g$ (i.e. composition of functions).
  \end{itemize}





\quad\\\hrule
\quad\\


\end{ex}

\begin{ex}
\begin{enumerate}
\item 

{\bf WeBWork}

\item 

{\bf WeBWork}

\item 

{\bf WeBWork}

\item 

{\bf WeBWork}

\item 

{\bf WeBWork}

\item 

{\bf WeBWork}

\item 

{\bf WeBWork}

\item 

{\bf WeBWork}

\item 

{\bf WeBWork}
\end{enumerate}
\end{ex}

\begin{eg}
The set $G = {\rm GL}(2, \mathbb{R})$
of real $2 \times 2$ matrices with nonzero determinants is a group
under matrix multiplication, with identity element:
\[e = \left(\begin{matrix} 1 &amp; 0\\0&amp; 1\end{matrix}\right).\]
In the group $G$, we have:
\[\left(\begin{matrix} a &amp; b \\ c &amp; d\end{matrix}\right)^{-1}
=
\frac{1}{ad - bc}\left(\begin{matrix}d &amp; -b \\-c &amp; a\end{matrix}\right)\]

Note that there are matrices $A, B \in {\rm GL}(2, \mathbb{R})$
such that $AB \neq BA$. Hence ${\rm GL}(2, \mathbb{R})$ is not abelian.





\quad\\\hrule
\quad\\


\end{eg}
\begin{ex}

The set ${\rm SL}(2, \mathbb{R})$ ( {\bf Special Linear Group} )
of real $2 \times 2$ matrices with determinant $1$ is a group under matrix multiplication.
\end{ex}


\begin{claim}
The identity element $e$ of a group $G$ is unique.
\end{claim}
\begin{proof}

Suppose there is an element $e' \in G$ such that $e' g = ge' = g$ for all $g \in G$.
Then, in particular, we have:
\[e'e = e\]
But since $e$ is an identity element, we also have $e'e = e'$. Hence, $e' = e$.
\end{proof}
\begin{claim}
Let $G$ be a group.
For all $g \in G$, its inverse $g^{-1}$ is unique.
\end{claim}
\begin{proof}

Suppose there exists $g' \in G$ such that $g'g = gg' = e$.
By the associativity of the group operation, we have:
\[g' = g'e = g'(g g^{-1}) = (g'g)g^{-1} = e g^{-1} = g^{-1}.\]
Hence, $g^{-1}$ is unique.
\end{proof}

Let $G$ be a group with identity element $e$.
For $g \in G$, $n \in \mathbb{N}$,
let:
\[\begin{split}
g^n &amp;:= \underbrace{g \cdot g \cdots g}_{n\text{ times}}.\\
g^{-n} &amp;:= \underbrace{g^{-1} \cdot g^{-1} \cdots g^{-1}}_{n\text{ times}}\\
g^0 &amp;:= e.
\end{split}\]
\begin{claim}

Let $G$ be a group.
\begin{enumerate}
\item 
For all $g \in G$,
we have:
\[(g^{-1})^{-1} = g.\]
  
\item 
For all $a, b \in G$,
we have:
\[(ab)^{-1} = b^{-1}a^{-1}.\]
  
\item 
For all $g \in G$, $n, m \in \mathbb{Z}$, we have:
\[g^n\cdot g^m = g^{n + m}.\]
  \end{enumerate}
\end{claim}
\begin{proof}

 {\bf Exercise.} 
\end{proof}

\begin{defn}
Let $G$ be a group, with identity element $e$.
The  {\bf order}  of $G$ is the number of elements in $G$.
The  {\bf order}  $\ord g$ of an  $g \in G$
is the smallest $n \in \mathbb{N}$ such that $g^n = e$.
If no such $n$ exists, we say that $g$ has  {\bf infinite order} .
\end{defn}
\begin{thm}

Let $G$ be a group with identity element $e$.
Let $g$ be an element of $G$. If $g^n = e$ for some $n \in \mathbb{N}$,
then $\ord g$ divides $n$.
\end{thm}
\begin{proof}

Shown in class.
\end{proof}

\end{document}