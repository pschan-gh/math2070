\documentclass[a4paper,12pt]{report}
\usepackage[T1]{fontenc}
\usepackage{times}
\usepackage{amsthm}
\usepackage{amscd,amssymb,stmaryrd}
\usepackage{amsmath}
\usepackage{graphicx}
\usepackage{fancybox}
\usepackage{amstext}
\usepackage{color}
\usepackage{mathtools}
\usepackage{pdflscape}
\usepackage{listings}
\usepackage{epic,eepic}
\usepackage{fancyhdr}
\usepackage{hyperref}
\usepackage[capitalise]{cleveref}


\newcommand{\abs}[1]{\left|#1\right|}
\newcommand{\ds}{\displaystyle}
\newcommand{\ol}[1]{\overline{#1}}
\newcommand{\oll}[1]{\overline{\overline{#1}}}
\newcommand{\bs}{\backslash}
\newcommand{\Frac}{\mathrm{Frac}}
\newcommand{\im}{\mathrm{im}\,}
\newcommand{\ZZ}{\mathbb{Z}}
\newcommand{\ra}{\longrightarrow}
\newcommand{\ord}{\mathrm{ord}\,}
\newcommand{\GL}{{\rm GL}}
\newcommand{\SL}{{\rm SL}}
\newcommand{\SO}{{\rm SO}}
\newcommand{\colvec}[1]{\begin{pmatrix}#1\end{pmatrix}}
\newcommand{\Span}{{\rm Span}\,}
\newcommand{\Rank}{{\rm Rank}\,}
\newcommand{\nullity}{{\rm nullity}\,}
\newcommand{\adj}{{\rm adj}\,}
\newcommand{\Proj}{{\rm Proj}}
\newcommand{\ora}{\overrightarrow}
\newcommand{\ve}{\varepsilon}
\newcommand{\phib}{\ol{\phi}}

\newcommand{\class}{}

\renewcommand{\ord}{\mathrm{ord}\,}

\newcounter{statement}
\numberwithin{statement}{chapter}

\newtheorem{thm}[statement]{Theorem}
\newtheorem{prop}[statement]{Proposition}
\newtheorem{defn}[statement]{Definition}
\newtheorem{lemma}[statement]{Lemma}
\newtheorem{claim}[statement]{Claim}
\newtheorem{cor}[statement]{Corollary}
\newtheorem{fact}[statement]{Fact}
\newtheorem{example}[statement]{\bf Example}
\newtheorem{eg}[statement]{\bf Example}
\newtheorem{ex}[statement]{\bf Exercise}
\newtheorem*{notation}{\bf Notation}
\newtheorem*{sol}{\bf Solution}
\newtheorem*{remark}{\bf Remark}
\numberwithin{equation}{chapter}
\numberwithin{section}{chapter}
\numberwithin{subsection}{section}

\renewcommand{\thesubsection}{\arabic{section}.\arabic{subsection}}
\renewcommand{\thesection}{\thechapter.\arabic{section}}
\begin{document}
\title{Math 2070}
\setcounter{chapter}{1}\setcounter{section}{0}
\setcounter{subsection}{0}

\numberwithin{equation}{chapter}

\chapter*{Math 2070 Week 1}
{\bf Topics: }Groups


\quad\\\hrule
\quad\\
\subsection*{Motivation}


\quad\\\hrule
\quad\\

\begin{itemize}
\item 
How many ways are there to color a cube, such that each face is either red or green?




 {\bf Answer:} 
10. Why?




\item 
How many ways are there to form a triangle with three sticks of equal lengths, colored
red, green and blue, respectively?




\item 
What are the symmetries of an equilateral triangle?




\textbf{Dihedral Group $D_3$}










\end{itemize}




\quad\\\hrule
\quad\\
\subsection*{Cayley Table}



\begin{center}
\begin{tabular}{||}
\hline
*&$a$&$b$&$c$ \\
\hline
$a$&$a^2$&$ab$&$ac$ \\
\hline
$b$&$ba$&$b^2$&$bc$ \\
\hline
$c$&$ca$&$cb$&$c^2$\\\hline
\end{tabular}
\end{center}



\quad\\\hrule
\quad\\





\textbf{Cayley Table for $D_3$}



\begin{center}
\begin{tabular}{||}
\hline

*
&$r_0$&$r_1$&$r_2$&$s_0$&$s_1$&$s_2$ \\
\hline
$r_0$&$r_0$&$r_1$&$r_2$&$s_0$&$s_1$&$s_2$ \\
\hline
$r_1$&$r_1$&$r_2$&$r_0$&$s_1$&$s_2$&$s_0$ \\
\hline
$r_2$&$r_2$&$r_0$&$r_1$&$s_2$&$s_0$&$s_1$ \\
\hline
$s_0$&$s_0$&$s_2$&$s_1$&$r_0$&$r_2$&$r_1$ \\
\hline
$s_1$&$s_1$&$s_0$&$s_2$&$r_1$&$r_0$&$r_2$ \\
\hline
$s_2$&$s_2$&$s_1$&$s_0$&$r_2$&$r_1$&$r_0$\\\hline
\end{tabular}
\end{center}



\quad\\\hrule
\quad\\
\section*{Groups}

\begin{defn}
A group $G$ is a set equipped with a binary operation $*: G \times G \ra G$ (typically called  {\bf  group operation}  or " {\bf multiplication} "), such that:
\begin{itemize}
\item 
 \underline{  {\bf Associativity} } 
\[(a* b)* c = a * (b * c),\]
for all $a, b, c \in G$. In other words, the group operation is  {\bf associative} .

\item 
 \underline{  {\bf Existence of an Identity Element} } 



There is an element $e \in G$, called an  {\bf identity element} , such that:
\[g* e = e * g = g,\]
for all $g \in G$.

\item 
 \underline{  {\bf Invertibility} } 



Each element $g \in G$ has an  {\bf inverse}  $g^{-1} \in G$, such that:
\[g^{-1}* g = g* g^{-1} = e.\]
\end{itemize}
\end{defn}

\begin{itemize}
\item 
Note that we do not require that $a* b = b * a$.
  
\item 
We often write $ab$ to denote $a* b$.
  \end{itemize}

\begin{defn}
If $ab = ba$ for all $a, b \in G$. We say that the group operation is
 {\bf commutative} , and that $G$ is an  {\bf abelian group} .
\end{defn}

\begin{eg}
The following sets are groups, with respect to the specified group operations:
\begin{itemize}
\item 
$G = \mathbb{Q} \backslash \{0\}$, where the group operation is the usual multiplication for rational numbers.
The identity is $e = 1$, and the inverse of $a \in \mathbb{Q}\backslash\{0\}$ is $a^{-1} = \frac{1}{a}$.
The group $G$ is abelian.
  
\item 
$G = \mathbb{Q}$, where the group operation is the usual addition $+$ for rational numbers. The identity is $e = 0$.
The inverse of $a \in \mathbb{Q}$ with respect to $+$ is $-a$.
Note that $\mathbb{Q}$ is  {\bf NOT}  a group with respect to multiplication. For in that case, we have $e = 1$, but $0 \in \mathbb{Q}$ has no inverse
$0^{-1} \in \mathbb{Q}$ such that $0\cdot 0^{-1} = 1$.
  \end{itemize}





\quad\\\hrule
\quad\\

\end{eg}

\begin{ex}
Verify that the following sets are groups under the specified binary operation:
\begin{itemize}
\item 
$(\mathbb{Z}, +)$
  
\item 
$(\mathbb{R}, +)$
  
\item 
$(\mathbb{R}^\times, \cdot)$
  
\item 
$(U_m, \cdot)$, where $m \in \mathbb{N}$,
\[U_m = \{1, \xi_m, \xi_m^2, \ldots, \xi_m^{m - 1}\},\]
and
$\xi_m = e^{2\pi i/m} = \cos(2\pi/m) + i\sin(2\pi/m) \in \mathbb{C}$.


  
\item 
The set of bijective functions $f: \mathbb{R} \ra \mathbb{R}$,
where $f * g:= f \circ g$ (i.e. composition of functions).
  \end{itemize}





\quad\\\hrule
\quad\\


\end{ex}

\begin{ex}
\begin{enumerate}
\item 

{\bf WeBWork}

\item 

{\bf WeBWork}

\item 

{\bf WeBWork}

\item 

{\bf WeBWork}

\item 

{\bf WeBWork}

\item 

{\bf WeBWork}

\item 

{\bf WeBWork}

\item 

{\bf WeBWork}

\item 

{\bf WeBWork}
\end{enumerate}
\end{ex}

\begin{eg}
The set $G = {\rm GL}(2, \mathbb{R})$
of real $2 \times 2$ matrices with nonzero determinants is a group
under matrix multiplication, with identity element:
\[e = \left(\begin{matrix} 1 &amp; 0\\0&amp; 1\end{matrix}\right).\]
In the group $G$, we have:
\[\left(\begin{matrix} a &amp; b \\ c &amp; d\end{matrix}\right)^{-1}
=
\frac{1}{ad - bc}\left(\begin{matrix}d &amp; -b \\-c &amp; a\end{matrix}\right)\]

Note that there are matrices $A, B \in {\rm GL}(2, \mathbb{R})$
such that $AB \neq BA$. Hence ${\rm GL}(2, \mathbb{R})$ is not abelian.





\quad\\\hrule
\quad\\


\end{eg}
\begin{ex}

The set ${\rm SL}(2, \mathbb{R})$ ( {\bf Special Linear Group} )
of real $2 \times 2$ matrices with determinant $1$ is a group under matrix multiplication.
\end{ex}


\begin{claim}
The identity element $e$ of a group $G$ is unique.
\end{claim}
\begin{proof}

Suppose there is an element $e' \in G$ such that $e' g = ge' = g$ for all $g \in G$.
Then, in particular, we have:
\[e'e = e\]
But since $e$ is an identity element, we also have $e'e = e'$. Hence, $e' = e$.
\end{proof}
\begin{claim}
Let $G$ be a group.
For all $g \in G$, its inverse $g^{-1}$ is unique.
\end{claim}
\begin{proof}

Suppose there exists $g' \in G$ such that $g'g = gg' = e$.
By the associativity of the group operation, we have:
\[g' = g'e = g'(g g^{-1}) = (g'g)g^{-1} = e g^{-1} = g^{-1}.\]
Hence, $g^{-1}$ is unique.
\end{proof}

Let $G$ be a group with identity element $e$.
For $g \in G$, $n \in \mathbb{N}$,
let:
\[\begin{split}
g^n &amp;:= \underbrace{g \cdot g \cdots g}_{n\text{ times}}.\\
g^{-n} &amp;:= \underbrace{g^{-1} \cdot g^{-1} \cdots g^{-1}}_{n\text{ times}}\\
g^0 &amp;:= e.
\end{split}\]
\begin{claim}

Let $G$ be a group.
\begin{enumerate}
\item 
For all $g \in G$,
we have:
\[(g^{-1})^{-1} = g.\]
  
\item 
For all $a, b \in G$,
we have:
\[(ab)^{-1} = b^{-1}a^{-1}.\]
  
\item 
For all $g \in G$, $n, m \in \mathbb{Z}$, we have:
\[g^n\cdot g^m = g^{n + m}.\]
  \end{enumerate}
\end{claim}
\begin{proof}

 {\bf Exercise.} 
\end{proof}

\begin{defn}
Let $G$ be a group, with identity element $e$.
The  {\bf order}  of $G$ is the number of elements in $G$.
The  {\bf order}  $\ord g$ of an  $g \in G$
is the smallest $n \in \mathbb{N}$ such that $g^n = e$.
If no such $n$ exists, we say that $g$ has  {\bf infinite order} .
\end{defn}
\begin{thm}

Let $G$ be a group with identity element $e$.
Let $g$ be an element of $G$. If $g^n = e$ for some $n \in \mathbb{N}$,
then $\ord g$ divides $n$.
\end{thm}
\begin{proof}

Shown in class.
\end{proof}
#course{Math 2070}
#week{2}
\title{Math 2070}
\setcounter{chapter}{2}\setcounter{section}{0}
\setcounter{subsection}{0}

\numberwithin{equation}{chapter}

\chapter*{Math 2070 Week 2}
{\bf Topics: }Groups


\begin{defn}
Let $G$ be a group, with identity element $e$.



The  {\bf order}  of $G$ is the number of elements in $G$.



The  {\bf order}  $\ord g$ of an  element $g \in G$
is the smallest $n \in \mathbb{N}$ such that $g^n = e$.
If no such $n$ exists, we say that $g$ has  {\bf infinite order} .
\end{defn}




\quad\\\hrule
\quad\\

\begin{thm}
\label{thm:orderdividesn}



Let $G$ be a group with identity element $e$.
Let $g$ be an element of $G$.  If $g^n = e$ for some $n \in \mathbb{N}$,
then $\ord g$ is finite, and moreover $\ord g$ divides $n$.
\end{thm}
\begin{proof}
Shown in class.
\end{proof}

\quad\\\hrule
\quad\\




\begin{ex}

If $G$ has finite order, then every element of $G$ has finite order.

\end{ex}




\begin{defn}
A group $G$ is  {\bf cyclic}  if there exists $g \in G$ such that every element of $G$ is equal to $g^n$
for some integer $n$.
In which case, we write: $G = \langle g \rangle$, and say that $g$ is a  {\bf generator}  of $G$.


Note: The generator of of a cyclic group might not be unique.

\end{defn}
\begin{eg}


$(U_m, \cdot)$ is cyclic.

\end{eg}
\begin{ex}

A finite cyclic group $G$ has order (i.e. size) $n$
if and only if each of its generators has order $n$.

\end{ex}
\begin{ex}

$(\mathbb{Q}, +)$ is not cyclic.

\end{ex}


\quad\\\hrule
\quad\\
\subsection*{Permutations}

\begin{defn}
Let $X$ be a set.  A  {\bf permutation}  of $X$ is a bijective map $\sigma : X \ra X$.
\end{defn}
\begin{claim}

The set $S_X$ of permutations of a set $X$
is a group with respect to $\circ$, the composition of maps.

\end{claim}
\begin{proof}

\begin{itemize}
\item 
Let $\sigma, \gamma$ be permutations of $X$.
By definition, they are bijective maps from $X$ to itself.
It is clear that $\sigma\circ\gamma$ is a bijective map from $X$ to itself,
hence $\sigma\circ\gamma$ is a permutation of $X$.  So $\circ$ is a well-defined
binary operation on $S_X$.

\item 
For $\alpha, \beta, \gamma \in S_X$, it is clear that
$\alpha\circ(\beta\circ \gamma) = (\alpha\circ\beta)\circ\gamma$.

\item 
Define a map $e : X \ra X$ as follows:
\[
e(x) = x,\quad \text{ for all } x \in X.
\]
It is clear that $e \in S_X$, and that $e \circ \sigma = \sigma\circ e = \sigma$
for all $\sigma \in S_X$.  Hence, $e$ is an identity element in $S_X$.

\item 
Let $\sigma$ be any element of $S_X$.  Since $\sigma : X \ra X$ is by assumption bijective,
there exists a bijective map $\sigma^{-1} : X \ra X$
such that $\sigma\circ\sigma^{-1} = \sigma^{-1}\circ \sigma = e$.
So $\sigma^{-1}$
is an inverse of $\sigma$ with respect to the operation $\circ$.
\end{itemize}


\end{proof}




 {\bf Terminology:} 
We call $S_X$ the  {\bf Symmetric Group}  on $X$.



 {\bf Notation:} 
If $X = \{1, 2, \ldots, n\}$, where $n \in \mathbb{N}$,
we denote $S_X$ by $S_n$.



For $n \in \mathbb{N}$, the group $S_n$ has $n!$ elements.



For $n \in \mathbb{N}$,
by definition an element of $S_n$ is a bijective map $\sigma : X \ra X$,
where $X = \{1, 2, \ldots, n\}$.
We often describe $\sigma$ using the following notation:
\[
\sigma = \left(\begin{matrix}
1 & 2 & \cdots & n\\
\sigma(1) & \sigma(2) & \ldots & \sigma(n)
\end{matrix}
\right)
\]

\begin{eg}
In $S_3$,
\[
\sigma = \left(
\begin{matrix}
1 & 2 & 3\\
3 & 2 & 1
\end{matrix}
\right)
\]
is the permutation on $\{1, 2, 3\}$
which sends $1$ to $3$, $2$ to itself, and $3$ to $1$,
i.e. $\sigma(1) = 3, \sigma(2) = 2, \sigma(3) = 1$.




For $\alpha, \beta \in S_3$ given by:
\[
\alpha =
\left(
\begin{matrix}
1 & 2 & 3\\
2 & 3 & 1
\end{matrix}\right),
\quad
\beta =
\left(\begin{matrix}
1 & 2 & 3\\
2 & 1 & 3
\end{matrix}\right),
\]
we have:




\[
\alpha\beta = \alpha\circ\beta
=
\left(\begin{matrix}
1 & 2 & 3\\
2 & 3 & 1
\end{matrix}\right) \circ
\left(\begin{matrix}
1 & 2 & 3\\
2 & 1 & 3
\end{matrix}\right)
=\left(\begin{matrix}
1 & 2 & 3\\
3 & 2 & 1
\end{matrix}\right)
\]
(since, for example, $\alpha\circ\beta: 1 \xmapsto{\beta} 2 \xmapsto{\alpha} 3$.).



We also have:




\[
\beta\alpha = \beta\circ\alpha
= \left(\begin{matrix}
1 & 2 & 3\\
2 & 1 & 3
\end{matrix}\right) \circ
\left(\begin{matrix}
1 & 2 & 3\\
2 & 3 & 1
\end{matrix}\right)
=
\left(\begin{matrix}
1 & 2 & 3\\
1 & 3 & 2
\end{matrix}\right)
\]

Since $\alpha\beta \neq \beta\alpha$, the group $S_3$ is non-abelian.




In general, for $n > 2$, the group $S_n$ is non-abelian ( {\bf Exercise:}  Why?).




For the same $\alpha \in S_3$ defined above, we have:
\[
\alpha^2 = \alpha\circ\alpha =
\left(\begin{matrix}
1 & 2 & 3\\
2 & 3 & 1
\end{matrix}\right)\circ
\left(\begin{matrix}
1 & 2 & 3\\
2 & 3 & 1
\end{matrix}\right) =
\left(\begin{matrix}
1 & 2 & 3\\
3 & 1 & 2
\end{matrix}\right)
\]
and:




\[
\alpha^3 = \alpha\cdot\alpha^2
= \left(\begin{matrix}
1 & 2 & 3\\
2 & 3 & 1
\end{matrix}\right)
\circ
\left(\begin{matrix}
1 & 2 & 3\\
3 & 1 & 2
\end{matrix}\right)
=
\left(\begin{matrix}
1 & 2 & 3\\
1 & 2 & 3
\end{matrix}\right) = e
\]

Hence, the order of $\alpha$ is $3$.

\end{eg}

\quad\\\hrule
\quad\\
\subsection*{Dihedral Group}

Consider the subset $\mathcal{T}$ of transformations of $\mathbb{R}^2$,
consisting of all rotations by fixed angles about the origin, and all reflections over lines through the origin.




Consider a regular polygon $P$ with $n$ sides in $\mathbb{R}^2$, centered at the origin.
Identify the polygon with its $n$ vertices, which form a subset $P = \{x_1, x_2, \ldots, x_n\}$
of $\mathbb{R}^2$.  If $\tau(P) = P$ for some $\tau \in \mathcal{T}$, we say that $P$ is  {\bf symmetric} 
with respect to $\tau$.




Intuitively, it is clear that $P$ is symmetric with respect to $n$ rotations $\{r_0, r_1,\ldots, r_{n - 1}\}$,
and $n$ reflections $\{s_1, s_2,\ldots, s_n\}$ in $\mathcal{T}$.









By Jim.belk - Own work, Public Domain, Link







\begin{thm}

The set $D_n := \{r_0, r_1,\ldots, r_{n - 1}, s_1, s_2,\ldots, s_n\}$ is a group,
with respect to the group operation defined by $\tau*\gamma = \tau\circ\gamma$
(composition of transformations).

\end{thm}
 {\bf Terminology:} 

$D_n$ is called a  {\bf dihedral group} .





\subsubsection*{More on $S_n$}




Consider the following element in $S_6$:
\[
\sigma = \left(
\begin{matrix}
1&2&3&4&5&6\\
5&4&3&6&1&2
\end{matrix}
\right)
\]

We may describe the action of $\sigma : \{1, 2, \ldots, 6\} \ra \{1, 2, \ldots, 6\}$
using the notation:
\[
\sigma = (15)(246),
\]

where $(n_1 n_2\cdots n_k)$ represents the permutation:
\[
n_1 \mapsto n_2 \dots n_i \mapsto n_{i + 1} \dots \mapsto n_k \mapsto n_1
\]

Viewing permutations as bijective maps,
the "multiplication" $(15)(246)$ is by definition the composition $(15)\circ(246)$.




We call $(n_1n_2\cdots n_k)$ a  {\bf $k$-cycle} .
Note that $3$ is missing from $(15)(246)$.
This corresponds to the fact that $3$ is fixed by $\sigma$.







\begin{claim}
Every non-identity permutation in $S_n$ is either a cycle or a product of disjoint cycles.
\end{claim}
\begin{proof}
Discussed in class.
\end{proof}




\quad\\\hrule
\quad\\

\begin{ex}

Disjoint cycles commute with each other.

\end{ex}




A 2-cycle is often called a  {\bf transposition} ,
for it switches two elements with each other.



\begin{claim}
Each element of $S_n$ is a product of (not necessarily disjoint) transpositions.
\end{claim}



Sketch of proof:


Show that each permutation not equal to the identity is a product of cycles,
and that each cycle is a product of transpositions:
\[
(a_1a_2\ldots a_k) = (a_1 a_k) (a_1 a_{k - 1})\cdots(a_1 a_3)(a_1 a_2)
\]

\begin{eg}


\[
\begin{split}
\left(
\begin{matrix}
1&2&3&4&5&6\\
5&4&3&6&1&2
\end{matrix}
\right) &= 
\class{steps1 steps}{(15)(246)}
\\&
\class{steps1 steps}{= (15)(26)(24)}
\\&
\class{steps1 steps}{ = (15)(46)(26)}
\end{split}
\]


\end{eg}




Note that a given element $\sigma$ of $S_n$
may be expressed as a product of transpositions in different ways,
but:
\begin{claim}
In every factorization of $\sigma$ as a product of transpositions,
the number of factors is either always even or always odd.
\end{claim}



\begin{proof}

 {\bf Exercise.} 
One approach: Show that there is a unique $n \times n$ matrix, with either $0$ or $1$ as its coefficients,
which sends each standard basis vector $\vec{e}_i$ in $\mathbb{R}^n$ to $\vec{e}_{\sigma(i)}$.
Then, use the fact that the determinant of the matrix corresponding to a transposition is $-1$,
and that the determinant function of matrices is multiplicative.
\end{proof}

\begin{ex}
\begin{enumerate}
\item  
{\bf WeBWork}

\item  
{\bf WeBWork}

\item  
{\bf WeBWork}

\item  
{\bf WeBWork}
\end{enumerate}\end{ex}\title{Math 2070}
\setcounter{chapter}{3}\setcounter{section}{0}
\setcounter{subsection}{0}

\numberwithin{equation}{chapter}

\chapter*{Math 2070 Week 3}
{\bf Topics: }Subgroups, Left Cosets, Index




\quad\\\hrule
\quad\\
\section*{Subgroups}

\begin{defn}
Let $G$ be a group.
A subset $H$ of $G$ is a  {\bf subgroup}  of $G$ if it satisfies the following properties:




\begin{itemize}
\item 
 {\bf Closure}  If $a, b \in H$, then $ab \in H$.

\item 
 {\bf Identity}  The identity element of $G$ lies in $H$.

\item 
 {\bf Inverses}  If $a \in H$, then $a^{-1} \in H$.
\end{itemize}

\end{defn}

In particular, a subgroup $H$ is a group with respect to the group operation on $G$,
and the identity element of $H$ is the identity element of $G$.





\begin{eg}
\label{eg:subgroups}


\begin{itemize}
\item 
For any $n \in \mathbb{Z}$,
$n\mathbb{Z}$ is a subgroup of $(\mathbb{Z}, +)$.

\item 
$\mathbb{Q}\bs\{0\}$ is a subgroup of $(\mathbb{R}\bs\{0\}, \cdot)$.

\item 
${\rm SL}(2, \mathbb{R})$ is a subgroup of ${\rm GL}(2, \mathbb{R})$.

\item 
The set of all rotations (including the trivial rotation) in a dihedral group $D_n$
is a subgroup of $D_n$.

\item 
Let $n \in \mathbb{N}$, $n \geq 2$.  
We say that $\sigma \in S_n$ is an  {\bf even permutation} 
if it is equal to the product of an even number of transpositions.
The subset $A_n$ of $S_n$ consisting of even permutations is a subgroup of $S_n$.
$A_n$ is called an  {\bf alternating group} .
\end{itemize}
\end{eg}

\begin{claim}
A subset $H$ of a group $G$ is a subgroup of $G$ if and only if $H$ is nonempty
and, for all $x, y \in H$, we have $xy^{-1} \in H$.
\end{claim}
\begin{proof}

Suppose $H \subseteq G$ is a subgroup.
Then, $H$ is nonempty since $e_G \in H$.
For all $x, y \in H$, we have $y^{-1} \in H$;
hence, $xy^{-1} \in H$.



Conversely, suppose $H$ is a nonempty subset of $G$,
and $xy^{-1} \in H$ for all $x, y \in H$.




\begin{itemize}
\item 
 {\bf Identity}  Let $e$ be the identity element of $G$.
Since $H$ is nonempty, it contains at least one element $h$.
Since $e = h \cdot h^{-1}$, and by hypothesis $h\cdot h^{-1} \in H$, the set $H$ contains $e$.

\item 
 {\bf Inverses} 
Since $e \in H$, for all $a \in H$ we have $a^{-1} = e\cdot a^{-1} \in H$.

\item 
 {\bf Closure} 
For all $a, b \in H$, we know that $b^{-1} \in H$.  Hence,
$ab = a\cdot(b^{-1})^{-1} \in H$.
\end{itemize}

Hence, $H$ is a subgroup of $G$.


\end{proof}

\begin{claim}
The intersection of two subgroups of a group $G$ is a subgroup of $G$.
\end{claim}
\begin{proof}
Exercise.
\end{proof}

\begin{thm}
Every subgroup of $(\mathbb{Z}, +)$ is cyclic.
\end{thm}
\begin{proof}

Let $H$ be a subgroup of $G = (\mathbb{Z}, +)$.
If $H = \{0\}$, then it is clearly cyclic.




Suppose $\abs{H} > 1$.  Consider the subset:
\[
S = \{h \in H \,:\, h > 0\} \subseteq H
\]
Since a subgroup is closed under inverse, 
and the inverse of any $z \in \mathbb{Z}$
with respect to $+$ is $-z$, the subgroup $H$ must contain at least one positive
element.  Hence, $S$ is a non-empty subset of $\mathbb{Z}$ bounded from below.




It then follows from the Least Integer Axiom that exists a minimum element $h_0$ in $S$.
That is $h_0 \leq h$ for any $h \in S$.




 {\bf Exercise.}  Show that $H = \langle h_0 \rangle$.



( {\bf Hint} : The Division Theorem for Integers could be useful here.)
\end{proof}
\begin{ex}
Every subgroup of a cyclic group is cyclic.
\end{ex}

\quad\\\hrule
\quad\\
\section*{Lagrange's Theorem}

Let $G$ be a group, $H$ a subgroup of $G$.
We are interested in knowing how large $H$ is relative to $G$.



We define a relation $\equiv$ on $G$ as follows:
\[
a \equiv b \text{ if } b = ah \text{ for some } h \in H,
\]
or equivalently:
\[
a \equiv b \text{ if } a^{-1}b \in H.
\]
 {\bf Exercise:}  $\equiv$ is an  {\bf equivalence relation} .



We may therefore partition $G$
into disjoint equivalence classes with respect to $\equiv$.
We call these equivalence classes the  {\bf left cosets}  of $H$.



Each left coset of $H$ has the form $aH = \{ah \,|\, h \in H\}$.




We could likewise define  right cosets.  These sets are of the form $Hb$, $b \in G$.
In general,
the number of left cosets and right cosets, if finite, are equal to each other


\begin{eg}
Let $G = (\mathbb{Z}, +)$.
Let:
\[
H = 3\mathbb{Z} =
\{\ldots, -9, -6, -3, 0, 3, 6, 9, \ldots\}
\]
The set $H$ is a subgroup of $G$.
The left cosets of $H$ in $G$ are as follows:
\[
3\mathbb{Z}, 1 + 3\mathbb{Z}, 2 + 3\mathbb{Z},
\]
where $i + 3\mathbb{Z} := \{i + 3k : k \in \mathbb{Z}\}$.



In general, for $n \in \mathbb{Z}$,
the left cosets of $n\mathbb{Z}$ in $\mathbb{Z}$ are:
\[
i + n\mathbb{Z}, \quad i = 0, 1, 2, \ldots, n - 1.
\]
\end{eg}

\begin{defn}
The number of left cosets of a subgroup $H$ of $G$ is called the  {\bf index}  of $H$ in $G$.
It is denoted by:
\[
[G : H]
\]
\end{defn}
\begin{eg}
Let $n \in \mathbb{N}$, 
$G = (\mathbb{Z}, +)$, $H = (n\mathbb{Z}, +)$.
Then,
\[
[G:H] = n.
\]
\end{eg}

\begin{eg}
Let $G = {\rm GL}(n, \mathbb{R})$.  Let:
\[
H = {\rm GL}^+(n, \mathbb{R}) := \left\{ h \in G : \det h > 0\right\}.
\]
( {\bf Exercise:}  $H$ is a subgroup of $G$.)



Let:
\[
s = \left(
\begin{matrix}
-1 & 0 & 0& 0 & 0\\
0&1 &0 &0 &0\\
0&0 &1 &0 &0\\
0&0 &0 &\ddots&0\\
0& 0 &0 &0 &1
\end{matrix}
\right) \in G
\]
Note that $\det s = \det s^{-1} = -1$.



For any $g \in G$, either $\det g > 0$ or $\det g < 0$.
If $\det g > 0$, then $g \in H$.
If $\det g < 0$, we write:
\[
g = (ss^{-1}) g = s(s^{-1} g).
\]
Since $\det s^{-1}g = (\det s^{-1})(\det g) > 0$, we have $s^{-1}g \in H$.
So, $G = H \sqcup s H$, and $[G : H] = 2$.
Notice that both $G$ and $H$ are infinite groups, but the index of $H$ in $G$ is finite.
\end{eg}



\begin{eg}
Let $G = {\rm GL}(n, \mathbb{R})$, $H = {\rm SL}(n, \mathbb{R})$.
For each $x \in \mathbb{R}^\times$, let:
\[
s_x = \left(
\begin{matrix}
x & 0 & 0& 0 & 0\\
0&1 &0 &0 &0\\
0&0 &1 &0 &0\\
0&0 &0 &\ddots&0\\
0& 0 &0 &0 &1
\end{matrix}
\right) \in G
\]
Note that $\det s_x = x$.



For each $g \in G$, we have:
\[
g = s_{\det g}{(s_{\det g}^{-1} g)} \in s_{\det g}H
%% g = s_{\det g}\underbrace{(s_{\det g}^{-1} g)}_{\in H}
\]
Moreover, for distinct $x, y \in \mathbb{R}^\times$, we have:
\[
\det (s_x^{-1}s_y) = y/x \neq 1.
\]
This implies that $s_x^{-1} s_y \notin H$, hence $s_yH$ and $s_xH$ are disjoint cosets.
We have therefore:
\[
G = \bigsqcup_{x \in \mathbb{R}^\times} s_x H.
\]
The index $[G: H]$ in this case is infinite.
\end{eg}








\begin{thm}
( {\bf Lagrange's Theorem} )
Let $G$ be a finite group.  Let $H$ be subgroup of $G$, then $\abs{H}$ divides $\abs{G}$.
More precisely, $\abs{G} = [G : H]\cdot\abs{H}$.
\end{thm}



We already know that the left cosets of $H$ partition $G$.
That is:
\[
G = a_1 H \sqcup a_2 H \sqcup \ldots \sqcup a_{[G:H]}H,
\]
where $a_i H \cap a_j H = \emptyset$ if $i \neq j$.
Hence, $\abs{G} = \sum_{i = 1}^{[G:H]} \abs{a_i H}$.

The theorem follows if we show that the size of each left coset of $H$ is equal to $\abs{H}$.



For each left coset $S$ of $H$, pick an element $a \in S$, and  define a map $\psi : H \ra S$ as follows:
\[
\psi(h) = ah.
\]
We want to show that $\psi$ is bijective.



For any $s \in S$,
by definition of a left coset (as an equivalence class) we have $s = ah$ for some $h \in H$.
Hence, $\psi$ is surjective.
If $\psi(h') = ah' = ah = \psi(h)$ for some $h', h \in H$,
then $h' = a^{-1}ah' = a^{-1}ah = h$.  Hence, $\psi$ is one-to-one.



So we have a bijection between two finite sets.  Hence, $\abs{S} = \abs{H}$.




\begin{cor}
Let $G$ be a finite group.
The order of every element of $G$  divides the order of $G$.
\end{cor}



Since $G$ is finite, any element of $g \in G$ has finite order $\ord g$.
Since the order of the subgroup:
\[
H = \langle g \rangle
= \{e, g, g^2, \ldots, g^{(\ord g) - 1}\}
\]
is equal to $\ord g$,
it follows from Lagrange's Theorem that $\ord g = \abs{H}$ divides $\abs{G}$.



\begin{cor}
If the order of a group $G$ is prime, then $G$ is a cyclic group.
\end{cor}



#course{Math 2070}
#week{4}
#topic{Groups}
\title{Math 2070}
\setcounter{chapter}{4}\setcounter{section}{0}
\setcounter{subsection}{0}

\numberwithin{equation}{chapter}

\chapter*{Math 2070 Week 4}
{\bf Topics: }Generators, Group Homomorphisms



\begin{thm}[Lagrange's Theorem]

\label{lagrangethm}



Let $G$ be a finite group. Let $H$ be subgroup of $G$, then $\abs{H}$ divides $\abs{G}$.
More precisely, $\abs{G} = [G: H]\cdot\abs{H}$.
\end{thm}
\begin{proof}

We already know that the left cosets of $H$ partition $G$.
That is:
\[G = a_1 H \sqcup a_2 H \sqcup \ldots \sqcup a_{[G:H]}H,\]
where $a_i H \cap a_j H = \emptyset$ if $i \neq j$.
Hence, $\abs{G} = \sum_{i = 1}^{[G:H]} \abs{a_i H}$.





The theorem follows if we show that the size of each left coset of $H$ is equal to $\abs{H}$.




For each left coset $S$ of $H$, pick an element $a \in S$, and define a map
$\psi: H \ra S$ as follows:
\[\psi(h) = ah.\]

We want to show that $\psi$ is bijective.




For any $s \in S$,
by definition of a left coset (as an equivalence class) we have $s = ah$ for some $h \in H$.
Hence, $\psi$ is surjective.




If $\psi(h') = ah' = ah = \psi(h)$ for some $h', h \in H$,
then $h' = a^{-1}ah' = a^{-1}ah = h$. Hence, $\psi$ is one-to-one.




So we have a bijection between two finite sets. Hence, $\abs{S} = \abs{H}$.

\end{proof}

\begin{cor}
Let $G$ be a finite group.
The order of every element of $G$ divides the order of $G$.
\end{cor}
Since $G$ is finite, any element of $g \in G$ has finite order $\ord g$.
Since the order of the subgroup:
\[H = \langle g \rangle
= \{e, g, g^2, \ldots, g^{(\ord g) - 1}\}\]
is equal to $\ord g$,
it follows from Lagrange's Theorem that $\ord g = \abs{H}$ divides $\abs{G}$.
\begin{cor}

If the order of a group $G$ is prime, then $G$ is a cyclic group.
\end{cor}
\begin{cor}

If a group $G$ is finite, then for all $g \in G$ we have:
\[
g^{\abs{G}} = e.
\]
\end{cor}

\begin{cor}
Let $G$ be a  finite group. Then a nonempty subset $H$ of $G$
is a subgroup of $G$ if and only if it is closed under the group operation of $G$
(i.e. $ab \in H$ for all $a, b \in H$).
\end{cor}
\begin{proof}

It is easy to see that if $H$ is a subgroup,
then it is closed under the group operation.
The other direction is left as an  {\bf Exercise} .
\end{proof}

\begin{eg}
Let $n$ be an integer greater than $1$. The group $A_n$ of even permutations
on a set of $n$ elements (see \cref{eg:subgroups}) has order $\displaystyle \frac{n!}{2}$.
\end{eg}
\begin{proof}

View $A_n$ as a subgroup of $S_n$, which has order $n!$.



 {\bf Exercise} : Show that $S_n = A_n\, \sqcup\, (12)A_n$.




Hence, we have $[S_n: A_n] = 2$.




It now follows from Lagrange's Theorem (\cref{lagrangethm}) that:
\[\abs{A_n} = \frac{\abs{S_n}}{[S_n: A_n]} = \frac{n!}{2}.\]
\end{proof}

\quad\\\hrule
\quad\\
\section*{Generators}

Let $G$ be a group, $X$ a nonempty subset of $G$.
The subset of $G$ consisting of elements of the form:
\[g_1^{m_1}g_2^{m_2}\cdots g_n^{m_n},
\quad\text{where}\quad
n \in \mathbb{N}, g_i \in X, m_i \in \mathbb{Z},\]
is a subgroup of $G$.
We say that it is the subgroup of $G$  {\bf generated}  by $X$.
If $X = \{x_1, x_2, \ldots, x_l\}$, $l \in \mathbb{N}$.
We often write:
\[\langle x_1, x_2, \ldots, x_l\rangle\]
to denote the subgroup generated by $X$.
\begin{eg}

In $D_n$, $\{ r_0, r_1, \ldots, r_{n - 1}\} = \langle r_1 \rangle$.
\end{eg}

If there exists a finite number of elements $x_1, x_2, \ldots, x_l \in G$
such that $G = \langle x_1, x_2, \ldots, x_l\rangle$,
we say that $G$ is  {\bf finitely generated} .




For example, every cyclic group is finitely generated, for it is generated by one
element.




Every finite group is finitely generated, since we may take the finite generating set
$X$ to be $G$ itself.

\begin{eg}
Consider $G = D_3$, and its subgroup $H = \{r_0, r_1, r_2\}$ consisting of its rotations.
(We use the convention that $r_k$ is the anticlockwise rotation by an angle of $2\pi k/3$).




By Lagrange's Theorem, the index of $H$ in $G$ is $[G: H] = \abs{G}/\abs{H} = 2$.
This implies that $G = H \sqcup gH$ for some $g \in G$.
Since $gH = H$ if $g \in H$, we may conclude that $g \notin H$. So, $g$ is a reflection.




Conversely, for any reflection $s \in D_3$, the left coset $sH$ is disjoint from $H$.
We have therefore $G = H {\sqcup} s_1H = H {\sqcup} s_2H = H {\sqcup} s_3H$,
which implies that $s_1 H = s_2 H = s_3 H$.




In particular, for a fixed $s = s_i$,
any element in $G$ is either a rotation or equal to $s r_i$ for some rotation $r_i$.
Since $H$ is a cyclic group, generated by the rotation $r_1$, we have $D_3 = \langle r_1, s \rangle$,
where $s$ is any reflection in $D_3$.
\end{eg}

\begin{ex}
\begin{enumerate}
\item 

{\bf WeBWork}

\item 

{\bf WeBWork}

\item 

{\bf WeBWork}

\item 

{\bf WeBWork}

\item 

{\bf WeBWork}

\item 

{\bf WeBWork}

\item 

{\bf WeBWork}

\item 

{\bf WeBWork}

\item 

{\bf WeBWork}

\item 

{\bf WeBWork}

\item 

{\bf WeBWork}

\item 

{\bf WeBWork}
\end{enumerate}
\end{ex}
\quad\\\hrule
\quad\\
\section*{Group Homomorphisms}

\begin{defn}
Let $G = (G, *)$, $G' = (G', *')$ be groups.
A  {\bf group homomorphism}  $\phi$ from $G$ to $G'$
is a map $\phi: G \ra G'$ which satisfies:
\[\phi(a * b) = \phi(a)*'\phi(b),\]
for all $a, b \in G$.
\end{defn}
\begin{claim}

If $\phi: G \ra G'$ is a group homomorphism, then:
\begin{enumerate}
\item 
$\phi(e_G) = e_{G'}$.

\item 
$\phi(g^{-1}) = \phi(g)^{-1}$, for all $g \in G$.

\item 
$\phi(g^n) = \phi(g)^n$, for all $g \in G$, $n \in \mathbb{Z}$.
\end{enumerate}
\end{claim}
\begin{proof}

We prove the first claim, and leave the rest as an exercise.
Since $e_G$ is the identity element of $G$, we have $e_G*e_G = e_G$.
On the other hand, since $\phi$ is a group homomorphism, we have:
\[\phi(e_G) =
\phi(e_G*e_G) = \phi(e_G)*'\phi(e_G).\]
Since $G'$ is a group, $\phi(e_G)^{-1}$ exists in $G'$, hence:
\[\phi(e_G)^{-1}*'\phi(e_G) = \phi(e_G)^{-1}*'(\phi(e_G)*'\phi(e_G))\]
The left-hand side is equal to $e_{G'}$, while by the associativity of $*'$
the right-hand side is equal to $\phi(e_G)$.

\end{proof}

Let $\phi: G \ra G'$ be a homomorphism of groups.
The image of $\phi$ is defined as:
\[\im \phi:= \phi(G):= \{\phi(g): g \in G\} \subseteq G'\]
The kernel of $\phi$ is defined as:
\[\ker \phi = \{g \in G: \phi(g) = e_{G'}\} \subseteq G.\]
\begin{claim}

The image of $\phi$ is a subgroup of $G'$.
The kernel of $\phi$ is a subgroup of $G$.
\end{claim}
\begin{claim}

A group homomorphism $\phi: G \ra G'$ is one-to-one if and only if $\ker \phi = \{e_G\}$.
\end{claim}

\begin{eg}[Examples of Group Homomorphisms]

\label{eg:grouphomomorphisms}



\begin{itemize}
\item 
$\phi: S_n \longrightarrow (\{\pm 1\}, \cdot)$,
\[\phi(\sigma) = \begin{cases} 1, & \sigma \text{ is an even permutation.}\\
-1, & \sigma \text{ is an odd permutation.}
\end{cases}\]

$\ker \phi = A_n$.


\item 
$\det: \GL(n, \mathbb{R}) \longrightarrow (\mathbb{R}^\times, \cdot)$




$\ker \det = \SL(n, \mathbb{R})$.


\item 
Let $G$ be the (additive) group of all real-valued continuous functions on $[0, 1]$.
\[\phi: G \longrightarrow (\mathbb{R}, +)\]
\[\phi(f) = \int_0^1 f(x)\,dx.\]

\item 
$\phi: (\mathbb{R}, +) \longrightarrow (\mathbb{R}^\times, \cdot)$.
\[\phi(x) = e^x.\]
\end{itemize}
\end{eg}

\begin{defn}
Let $G$, $G'$ be groups. A map $\phi: G \ra G'$ is a group  {\bf isomorphism}  if it is a bijective
group homomorphism.
\end{defn}

Note that if a homomorphism $\phi$ is bijective, then $\phi^{-1}: G' \ra G$ is also a homomorphism,
and consequently, $\phi^{-1}$ is an isomorphism.
If there exists an isomorphism between two groups $G$ and $G'$,
we say that the groups $G$ and $G'$ are  {\bf isomorphic} .

\begin{defn}
Fix an integer $n > 0$.



For any $k \in \mathbb{Z}$, let $\ol{k}$ denote the remainder of the division of $k$
by $n$.



Let $\mathbb{Z}_n = \{0, 1, 2, \ldots, n - 1\}$.
We define a binary operation $+_{\mathbb{Z}_n}$
on $\mathbb{Z}_n$ as follows:
\[k +_{\mathbb{Z}_n} l = \ol{k + l}.\]
\end{defn}
\begin{ex}

$\mathbb{Z}_n = (\mathbb{Z}_n, +_{\mathbb{Z}_n})$
is a group, with identity element $0$, and $j^{-1} = n - j$ for any $j \in \mathbb{Z}_n$.

\end{ex}

\begin{eg}
\label{eg:cycliczn}


Let $n > 2$.
Let $H = \{r_0, r_1, r_2, \ldots, r_{n - 1}\}$
be the subgroup of $D_n$ consisting of all rotations,
where $r_1$ denotes the anticlockwise rotation by the angle $2\pi/n$,
and $r_k = r_1^k$.
Then, $H$ is isomorphic to $\mathbb{Z}_n = (\mathbb{Z}_n, +_{\mathbb{Z}_n})$.
\end{eg}
\begin{proof}

Define $\phi: H \ra \mathbb{Z}_n$ as follows:
\[\phi(r_k) = k, \quad k \in \{0, 1, 2, \ldots, n - 1\}.\]




For any $k \in \mathbb{Z}$,
let $\ol{k} \in \{0, 1, 2, \ldots, n - 1\}$
denote the remainder of the division of $k$ by $n$.
By the Division Theorem for Integers, we have:
\[
k = nq + \ol{k}
\]
for some integer $q \in \mathbb{Z}$.




It now follows from $\ord r_1 = n$ that,
for all $r_i, r_j \in H$, we have:

\[\begin{split}
r_i r_j &= r_1^i r_1^j = r_1^{i + j}
\\&
\class{steps2 steps}{= r_1^{nq + \ol{i + j}}}
\\
&
\class{steps2 steps}{= \left(r_1^n\right)^q r_1^{\ol{i + j}}}
\\
&
\class{steps2 steps}{= r_{\ol{i + j}}.}
\end{split}\]





Hence,

\[\begin{split}
\phi(r_i r_j) &= \phi(r_{\ol{i + j}})
\\&
\class{steps3 steps}{= \ol{i + j}}
\\&
\class{steps3 steps}{= i +_{\mathbb{Z}_n} j}
\\&
\class{steps3 steps}{=\phi(r_i) +_{\mathbb{Z}_n} \phi(r_j).}
\end{split}\]


This shows that $\phi$ is a homomorphism.
It is clear that $\phi$ is surjective, which then implies that
$\phi$ is one-to-one, for the two groups have the same size.
Hence, $\phi$ is a bijective homomorphism, i.e. an isomorphism.


\end{proof}
#course{Math 2070}
#week{5}
#topic{Group Homomorphisms}
#topic{Rings}

$\newcommand{\ord}{\mathrm{ord}\,}$
\title{Math 2070}
\setcounter{chapter}{5}\setcounter{section}{0}
\setcounter{subsection}{0}

\numberwithin{equation}{chapter}

\chapter*{Math 2070 Week 5}
{\bf Topics: }Group Homomorphisms, Rings




\begin{claim}
Any cyclic group of finite order $n$ is isomorphic to $\mathbb{Z}_n$.
\end{claim}
\begin{proof}

Sketch of Proof:



By definition, a cyclic group $G$ is equal to $\langle g\rangle$ for some $g \in G$.
Moreover, $\ord g = \ord G$.



Define a map $\phi : G \longrightarrow \mathbb{Z}_n$ as follows:
\[
\phi(g^k) = k, \quad k \in \{0, 1, 2, \ldots, n - 1\}.
\]
Show that $\phi$ is a group isomorphism.



(For reference, see the discussion of \cref{eg:cycliczn}.)
\end{proof}



\begin{cor}

If $G$ and $G'$ are two finite cyclic groups of the same order, then $G$ is isomorphic to $G'$.

\end{cor}



\begin{ex}

An infinite cyclic group is isomorphic to $(\mathbb{Z}, +)$.
\end{ex}



\begin{ex}

Let $G$ be a cyclic group, then any group which is isomorphic to $G$ is also cyclic.
\end{ex}




\quad\\\hrule
\quad\\
\subsection*{Product Group}

 {\bf Product Group} 



Let $(A, \ast_A), (B, \ast_B)$ be groups.  The direct product:
\[
A \times B := \{(a, b)\;|\; a\in A, b \in B\}
\]
has a natural group structure 
where the group operation $\ast$ is defined as follows:
\[
(a, b)\ast(a', b') = (a\ast_A a', b\ast_B b'), \quad (a, b), (a', b') \in A\times B. 
\]

The identity element of $A \times B$ is $e = (e_A, e_B)$, where $e_A, e_B$ 
are the identity elements of $A$ and $B$, respectively.




For any $(a, b) \in A \times B$, we have $(a, b)^{-1} = (a^{-1}, b^{-1})$,
where $a^{-1}$, $b^{-1}$ are the inverses of $a$, $b$ in the groups $A$, $B$, 
respectively.




For any collection of groups $A_1, A_2, \ldots, A_n$,
we may similarly define a group operation $\ast$ on:
\[
A_1 \times A_2 \times \cdots \times A_n
:= \{(a_1, a_2, \ldots, a_n)\;|\;a_i \in A_i, i = 1, 2, \ldots n\}.
\]
That is:
\[
(a_1, a_2, \ldots, a_n)\ast (a_1', a_2', \ldots, a_n')
=
(a_1\ast_{A_1} a_1', a_2\ast_{A_2} a_2', \ldots, a_n \ast_{A_n} a_n')
\]
The identity element of $A_1 \times A_2 \times \cdots \times A_n$ is:
\[
e = (e_{A_1}, e_{A_2}, \ldots, e_{A_n}).
\]
For any $(a_1, a_2, \ldots, a_n) \in A_1 \times A_2 \times \cdots \times A_n$, 
its inverse is:
\[
(a_1, a_2, \ldots, a_n)^{-1} = (a_1^{-1}, a_2^{-1}, \ldots, a_n^{-1}).
\]


\begin{ex}
$\mathbb{Z}_6$ is isomorphic to $\mathbb{Z}_2\times\mathbb{Z}_3$.
\end{ex}
\begin{proof}

 {\bf Hint:} 



Show that $\mathbb{Z}_2\times\mathbb{Z}_3$ is a cyclic group generated by $(1, 1)$.
\end{proof}
\begin{eg}

The cyclic group $\mathbb{Z}_4$ is not isomorphic to $\mathbb{Z}_2 \times \mathbb{Z}_2$.
\end{eg}
\begin{proof}

Each element of $G = \mathbb{Z}_2\times\mathbb{Z}_2$ is of order at most $2$.
Since $\abs{G} = 4$, $G$ cannot be generated by a single element.  
Hence, $G$ is not cyclic,
so it cannot be isomorphic to the cyclic group $\mathbb{Z}_4$.


\end{proof}




\begin{ex}
Let $G$ be an abelian group, then any group which is isomorphic to $G$ is abelian.
\end{ex}
\begin{eg}

The group $D_6$ has $12$ elements.  We have seen that $D_6 = \langle r_1, s\rangle$, where $r_1$ is a rotation of order $6$,
and $s$ is a reflection, which has order $2$.  So, it is reasonable to ask if $D_6$ is isomorphic to $\mathbb{Z}_6 \times \mathbb{Z}_2$.  The answer is no.
For $\mathbb{Z}_6 \times \mathbb{Z}_2$ is abelian, but $D_6$ is not.

\end{eg}




\begin{claim}
The dihedral group $D_3$ is isomorphic to the symmetric group $S_3$.
\end{claim}
\begin{proof}

We have seen that $D_3 = \langle r, s\rangle$, where $r = r_1$ and 
$s$ is any fixed reflection,
with:
\[
\ord r = 3,\quad \ord s = 2,\quad srs = r^{-1}.
\]
In particular , any element in $D_3$ may be expressed as $r^is^j$,
with $i \in \{0, 1, 2\}$, $j \in \{0, 1\}$.




We have also seen that $S_3 = \langle a, b \rangle$, 
where:
\[
a = (123),\quad b = (12),\quad \ord a = 3,\quad \ord b = 2, \quad bab = a^{-1}.
\]
Hence, any element in $S_3$ may be expressed as $a^ib^j$,
with $i \in \{0, 1, 2\}$, $j \in \{0, 1\}$.




Define map $\phi: D_3 \longrightarrow S_3$ as follows:
\[
\phi(r^is^j) = a^ib^j, \quad i, j \in \mathbb{Z}
\]




We first show that $\phi$ is well-defined:
That is, whenever $r^i s^j = r^{i'}s^{j'}$,
we want to show that:
\[
\phi(r^i s^j) = \phi(r^{i'}s^{j'}).
\]

The condition $r^i s^j = r^{i'}s^{j'}$ implies that:
\[
r^{i - i'} = s^{j' - j}
\]

This holds only if  $r^{i - i'} = s^{j' - j} = e$,
since no rotation is a reflection.




Since $\ord r = 3$ and $\ord s = 2$, we have:
\[
3 | (i - i'), \quad 2 | (j' - j),
\]
by \cref{thm:orderdividesn}.




Hence,

\begin{align*}
\phi(r^is^j) \phi(r^{i'}s^{j'})^{-1} 
&= (a^ib^j)(a^{i'}b^{j'})^{-1}
&
\\
&
\class{steps4 steps}{= a^i b^jb^{-j'} a^{-i'}} 
&
\\
&
\class{steps4 steps}{= a^i b^{j - j'} a^{-i'}} 
&
\\
&
\class{steps4 steps}{= a^{i - i'}} 
&
\class{steps4 steps}{\text{ since } \ord b = 2.}
\\
&
\class{steps4 steps}{ = e} 
&
\class{steps4 steps}{\text{ since } \ord a = 3.}
\end{align*}

This implies that 
$\phi(r^is^j) = \phi(r^{i'}s^{j'})$.
We conclude that $\phi$ is well-defined.




We now show that $\phi$ is a group homomorphism:




Given $\mu, \mu' \in \{0, 1, 2\}$, $\nu, \nu' \in \{0, 1\}$, we have:
\[
\phi(r^\mu s^\nu \cdot r^{\mu'}s^{\nu'}) = 
\begin{cases}
\phi(r^{\mu + \mu'}s^{\nu'}),&\text{if } \nu = 0;\\
\phi(r^{\mu - \mu'}s^{\nu + \nu'}),&\text{if } \nu = 1.
\end{cases}
\]

\[
=
\begin{cases}
a^{\mu + \mu'}b^{\nu'},&\text{if } \nu = 0;\\
a^{\mu - \mu'}b^{\nu + \nu'} = a^{\mu}b^{\nu}a^{\mu'}b^{\nu'},&\text{if } \nu = 1.
\end{cases}
\]

\[
=\phi(r^\mu s^\nu)\phi(r^{\mu'}s^{\nu'}).
\]
This shows that $\phi$ is a group homomorphism.




To show that $\phi$ is a group isomorphism,
it remains to show that it is surjective and one-to-one.



It is clear that $\phi$ is surjective.  We leave it as an exercise
to show that $\phi$ is one-to-one.


\end{proof}




\begin{eg}
The group:
\[
G = \left\{g \in \mathrm{GL}(2, \mathbb{R}) \,\left|\,
g = \left(
\begin{matrix}
\cos \theta & -\sin \theta\\
\sin \theta & \cos \theta
\end{matrix}
\right) \text{ for some } \theta \in \mathbb{R}
\right.
\right\}
\]
is isomorphic to
\[
G' = \{z \in \mathbb{C} : \abs{z} = 1\}.
\]
Here, the group operation on $G$ is matrix multiplication, and the group operation on $G'$ is the multiplication of complex numbers.

\quad\\\hrule
\quad\\


Each element in $G'$ is equal to $e^{i\theta}$ for some $\theta \in \mathbb{R}$.
Define a map $\phi : G \ra G'$ as follows:
\[
\phi\left(\left(
\begin{matrix}
\cos \theta & -\sin \theta\\
\sin \theta & \cos \theta
\end{matrix}
\right)\right) = e^{i\theta}.
\]

 {\bf Exercise:}  
Show that $\phi$ is a well-defined map.
Then, show that it is a bijective group homomorphism.
\end{eg}









\quad\\\hrule
\quad\\
\section*{Rings}

\quad\\\hrule
\quad\\
\subsection*{Some Results in Elementary Number Theory}




\begin{thm}[Division Theorem]

\label{thm:divalg}



Let $a, b \in \mathbb{Z}$, $a \neq 0$,
then there exist unique $q$ (called the quotient), and $r$ ( {\bf remainder} ) in $\mathbb{Z}$, satisfying
$0 \leq r < \abs{a}$, such that $b = aq + r$.
\end{thm}
\begin{proof}

We will prove the case $a > 0$, $b \geq 0$.  The other cases are left as exercises.


Fix $a > 0$.
First, we prove the existence of the quotient $q$ and remainder $r$ for any $b \geq 0$,
using mathematical induction.


 {\bf The base step}  corresponds to the case $0 \leq b < a$.
In this case, if we let $q = 0$ and $r = b$, then indeed $b = qa + r$, 
where $0 \leq r  = b < a$.  Hence, $q$ and $r$ exist.


 {\bf The inductive step}  of the proof of the existence of $q$ and $r$ is as follows: 
Suppose the existence of the quotient and remainder holds for all non-negative $b' < b$,
we want to show that it must also hold for $b$.  


First, we may assume that
$b \geq a$, since the case $b < a$ has already been proved.  Let $b' = b - a$.
Then, $0 \leq b' < b$,
so by the inductive hypothesis we have $b' = q'a + r'$ 
for some $q', r' \in \mathbb{Z}$ such that $0 \leq r' < a$.


This implies that $b = b' + a = (q' + 1)a + r'$.  


So, if we let $q = q'+ 1$
and $r = r'$, then $b = qa + r$, where $0 \leq r < a$.
This establishes the existence of $q, r$ for $b$.  
Hence, by mathematical induction, the existence of $q, r$ holds for all $b \geq 0$.


Now we prove the uniqueness of $q$ and $r$.
Suppose $b = qa + r = q'a + r'$, where $q, q', r, r' \in \mathbb{Z}$, 
with $0 \leq r, r' < a$.


Then, $qa+ r = q'a+r'$ implies that $r - r' = (q' - q)a$.
Since $0 \leq r, r' < a$, we have:


\[
a > \abs{r - r'} = \abs{q' - q} a.
\]


Since $q' - q$ is an integer, the above inequality implies that $q'  - q = 0$,
i.e. $q' = q$, which then also implies that $r' = r$.
We have therefore established the uniqueness of $q$ and $r$.



The proof of the theorem, for the case $a > 0, b \geq 0$, is now complete.


\end{proof}







\begin{defn}
The  {\bf Greatest Common Divisor}  $gcd(a, b)$ of $a, b \in \mathbb{Z}$ is the largest positive integer
$d$ which divides both $a$ and $b$ (Notation: $d | a$ and $d | b$).
\end{defn}



 {\bf Note.} 

If $a \neq 0$, then $gcd(a, 0) = \abs{a}$.
$gcd(0, 0)$ is undefined.




\begin{lemma}

If $b = aq + r$ $(a, b, q, r \in \mathbb{Z})$, then
$gcd(b, a) = gcd (a, r)$.
\end{lemma}
\begin{proof}

If $d | a$ and $d | b$, then $d | r = b - aq$.
Conversely, if $d | a$ and $d | r$, then $d|a$ and $d|b = qa+r$.
So, the set of common divisors of $a, b$ is the same as the set of the common divisors of $a, r$.
If two finite sets of integers are the same, then their largest elements are clearly the same.
In other words:
\[
gcd(b, a) = gcd(a, r).
\]


\end{proof}




\quad\\\hrule
\quad\\
\subsection*{The Euclidean Algorithm}


Suppose $\abs{b} \geq \abs{a}$.  Let $b_0 = b$, $a_0 = a$.
Write $b_0 = a_0q_0 + r_0$, where $0 \leq r_0 < \abs{a_0}$.


For $n > 0$, let $b_n = a_{n - 1}$ and $a_n = r_{n - 1}$, where
$r_n$ is the remainder of the division of $b_n$ by $a_n$.  That is,
\[
b_n = a_nq_n + r_n, \quad  0 \leq r_n < \abs{a_n}.
\]

If $r_0 = 0$, then that means that $a | b$, and $gcd(a, b) = \abs{a}$.
Now, suppose $r_0 > 0$.
Since $r_n$ is a non-negative integer and  $0 \leq r_n < r_{n - 1}$,
eventually, $r_n = 0$ for some $n \in \mathbb{N}$.




\begin{claim}

$gcd(b, a) = \abs{a_n}$.

\end{claim}
\begin{proof}

By the previous lemma,

\[
\begin{split}
gcd(b, a) &= gcd(b_0, a_0)\\
 &
\class{steps5 steps}{= gcd(a_0, r_0) = gcd(b_1, a_1)}
\\
& 
\class{steps5 steps}{= gcd(a_1, r_1) = gcd(b_2, a_2)}
\\
& 
\class{steps5 steps}{= \ldots}
\\
& 
\class{steps5 steps}{
= gcd(a_{n}, r_{n}) = gcd(a_n, 0) = \abs{a_n}.
}
\end{split}
\]



\end{proof}




\begin{eg}
Find $gcd(285,255)$.





\[
\begin{split}
\underbrace{285}_{b_0} &= \underbrace{255}_{a_0} \underbrace{1}_{q_0} + \underbrace{30}_{r_0}
\\
\class{steps6 steps}{
\underbrace{255}_{b_1 = a_0}
}
&
\class{steps6 steps}{=\underbrace{30}_{a_1 = r_0} \underbrace{8}_{q_1} + \underbrace{15}_{r_1}}
\\
\class{steps6 steps}{\underbrace{30}_{b_2}} 
&
\class{steps6 steps}{=\underbrace{15}_{a_2} \underbrace{2}_{q_2} + \underbrace{0}_{r_2}}
\end{split}
\]


So, $gcd(285, 255) = r_1 = 15$.

\end{eg}





\end{document}